\documentclass{article}

% Remember to also load the algostyle.sty file into your project.
\usepackage{algostyle}

% Insert new packages here.

\begin{document}
\begin{question}
Alice is planting $n_1$ flowers $f_1, \dots, f_{n_1}$ among $n_2$ rectangular gardens $\mathcal G_1, \dots, \mathcal G_{n_2}$. Bob's task is to determine which flowers belong to which gardens. Alice informs Bob that no two gardens overlap; therefore, if a flower belongs to a garden, then the flower belongs to {\em exactly} one garden and a garden can contain multiple flowers. If a flower $f_i$ does not belong to any garden, then Bob returns an {\em undefined} garden for $f_i$.

Each garden $\mathcal G_i$ is given by a pair of points $\mathcal G_{BL}[i]$ and $\mathcal G_{TR}[i]$, representing the bottom left and top right corners of the garden respectively. Each flower is represented by a point $F[i]$ representing its location. Let $n = n_1 + n_2$.

\begin{figure}[H]
    \centering
    \begin{tikzpicture}

        \fill[applegreen!20, draw = black] (0, 0) rectangle (2, 1);
        \fill[applegreen!20, draw = black] (2.5, -0.2) rectangle (3.5, 1.5);
        \fill[applegreen!20, draw = black] (0.5, -0.8) rectangle (2.25, -0.2);
        \fill[applegreen!20, draw = black] (3, -0.5) rectangle (3.75, -2);
        
        \node[minimum size = 3pt, inner sep = 0pt] at (1, 0.5) {$\twemoji{sunflower}$};
        \node[minimum size = 3pt, inner sep = 0pt] at (3.25, 1) {$\twemoji{sunflower}$};
        \node[minimum size = 3pt, inner sep = 0pt] at (1.5, -0.5) {$\twemoji{sunflower}$};
        \node[minimum size = 3pt, inner sep = 0pt] at (1.2, -0.55) {$\twemoji{sunflower}$};
        \node[minimum size = 3pt, inner sep = 0pt] at (3.2, -1) {$\twemoji{sunflower}$};
    \end{tikzpicture}
    \caption{{\em A collection of $n_1 = 5$ flowers and $n_2 = 4$ gardens}.}
    % \label{fig:enter-label}
\end{figure}

Formally, you are given three arrays:
\begin{itemize}
    \item $\mathcal G_{BL} = [(x_1, y_1), \dots, (x_{n_2}, y_{n_2})]$, where $\mathcal G_{BL}[i] = (x_i, y_i)$ represents the bottom left point of garden $\mathcal G_i$,
    \item $\mathcal G_{TR} = [(x_1, y_1), \dots, (x_{n_2}, y_{n_2})]$, where $\mathcal G_{TR}[i] = (x_i, y_i)$ represents the top right point of garden $\mathcal G_i$, and
    \item $F = [(x_1, y_1), \dots, (x_{n_1}, y_{n_1})]$, where $F[i] = (x_i, y_i)$ represents the location of flower $f_i$.
\end{itemize}

You are not guaranteed that a flower belongs to a garden; it is possible that a flower is planted in none of the gardens. Your goal is to return the garden that a flower $f_i$ belongs to (if any), for {\em each} $f_i$.

\begin{itemize}
    \item For example, in the diagram above, if the flower ordering is sorted by its $x$-coordinate and the rectangular gardens are sorted by their $x$-coordinate of $\mathcal G_{BL}$, then we return \[f_1: \mathcal G_1, \quad f_2: \mathcal G_2, \quad f_3: \mathcal G_2, \quad f_4: \mathcal G_4, \quad f_5: \mathcal G_3.\]
\end{itemize}

\begin{enumerate}[label = (\alph*)]
    \item We first solve the special case where all of the gardens intersect with a horizontal line. Describe an $O(n \log n)$ algorithm to determine which flowers belong to which gardens (if such a garden exists).

    \begin{figure}[H]
        \centering
        \begin{tikzpicture}
            \fill[applegreen!20, draw = black] (0, 0) rectangle (2, 1);
            \fill[applegreen!20, draw = black] (2.5, -0.2) rectangle (3.5, 1.5);
            \fill[applegreen!20, draw = black] (3.75, 0.3) rectangle (4.5, 0.7);
            \fill[applegreen!20, draw = black] (4.75, -0.5) rectangle (5.75, 1.35);
    
            \draw[dotted] (-0.5, 0.5) -- (6.25, 0.5);
            % \draw (0.5, -0.8) rectangle (2.25, -0.2);
            % \draw (3, -0.5) rectangle (3.75, -2);
    
            \node[, minimum size = 3pt, inner sep = 0pt] at (1, 0.5) {$\twemoji{sunflower}$};
            \node[minimum size = 3pt, inner sep = 0pt] at (3.25, 1) {$\twemoji{sunflower}$};
            \node[minimum size = 3pt, inner sep = 0pt] at (4, 0.7) {$\twemoji{sunflower}$};
            \node[minimum size = 3pt, inner sep = 0pt] at (3.25, 0) {$\twemoji{sunflower}$};
            \node[minimum size = 3pt, inner sep = 0pt] at (1.2, 0.25) {$\twemoji{sunflower}$};
            \node[minimum size = 3pt, inner sep = 0pt] at (5, -0.1) {$\twemoji{sunflower}$};
        \end{tikzpicture}
        \caption{{\em A collection of $n_1 = 6$ flowers and $n_2 = 4$ gardens that intersect with a horizontal line.}}
        % \label{fig:enter-label}
    \end{figure}

    {\bfseries Hint.} {\em What do you know about two adjacent gardens if they have to intersect with a horizontal line?}

    \item We now remove the assumption that every garden intersects with a horizontal line. Describe an $O(n (\log n)^2)$ algorithm to determine which flowers belong to which gardens (if such a garden exists).
\end{enumerate}

{\bfseries Note.} {\em This was an assignment problem from COMP3121/9101, 23T3.}
\end{question}

\begin{rubric}
\begin{itemize}
    \item Your solution should clearly outline which subpart you're answering.

    \item If your solution is a simple modification of a previous solution, you do not need to restate the solution; you can simply refer to the previous part in your solution.

    \item As usual, you should argue the correctness of the algorithm and its time complexity.

    \item This task will form part of the portfolio.
    \item Ensure that your argument is clear and keep reworking your solutions until your lab demonstrator is happy with your work.
\end{itemize}
\end{rubric}

\begin{solution}
\begin{enumerate}[label = (\alph*)]
    \item We must only consider $x$ coordinates as no two gardens lined in a horizontal may intersect. We will denote such intervals as $(x_i^{(b)}, x_i^{(t)})$ as the garden interval for $\mathcal G_i$, and for flower $f_i$, we will use $f^{(x)}_i$. If for any flower $f_j^{(x)} \in (x_i^{(b)}, x_i^{(t)})$, then it must be so that $f_j \in \mathcal{G}_i$. This will be our check to verify if the flower is in the garden.

We begin by sorting our gardens by their intervals, which can be done by considering their $x^{(t)}_i$ and sorting based on those values. Then, for each flower, we perform a binary search on the gardens. If $f_j^{(x)} \in (x_i^{(b)}, x_i^{(t)})$, then we stop. If $f_j^{(x)}  \leq (x_i^{(b)}$, then we restrict our search space to the bottom half of the array and repeat. Similarly, if $f_j^{(x)}  \geq (x_i^{(t)}$, then we restrict the searching domain to the top half. If a garden has been found, one must then simply check the $y$ coordinates by seeing if $f_j^{(y)} \in (y_i^{(b)}, y_i^{(t)})$, where $f_j^{(y)}$ is the $y$ coordinate of $f_j$, and $y_i^{(b)}$ and $y_i^{(t)}$ is the bottom-left and top-right $y$ coordinates of $\mathcal{G}_i$ respectively.

Sorting the gardens is done in $O(n_2 \log n_2)$ time for the $n_2$ gardents and the binary search of the gardens for $n_1$ flowers would be $O(n_1 \log n_2)$, so our final process will run in $O\left( (n_1 + n_2) \log n_2\right) = O\left( n \log n\right)$.

    \item Our goal is to use the solution from part (a) by identifying horizontal lines that could work. Since we are only given numeric values, we can simply consider some measurment of central tendency, i.e. the mean or median. We will call this value $M$. Any garden passing through the line $y = M$ will be able to use the algorithm from part (a). If a garden is not cut by this line, it must then be either above or below the line. So we can isolate the gardens above and below and rerun the algorithm recursively. For any flower that is unaccounted for in this algorithm, we can divide into set of flowers that are above and below $y = M$, making the process more efficient for checking. We stop this process if we are left with only one element (a single flower or garden).

Our algorithm must run in $O(n \log n)$ each time we run part (a)'s algorithm. So, our time complexity will be recursively called as $R_n = 2R_{\lfloor n/2 \rfloor} + O(n \log n) \leq 2R_{\lfloor n/2 \rfloor} + Cn \log n$, where $C$ is some constant. We can solve this quite simply by verifying:$$R_n  \leq  2R_{\lfloor n/2 \rfloor} + Cn \log n\leq  2\left[ 2R_{\lfloor n/4 \rfloor} + Cn \log n\right] + C\dfrac{n}{2} \log \dfrac{n}{2} \leq 4R_{\lfloor n/4 \rfloor} + 2Cn\log n.$$ We must repeat this process $\log_2 (n)$ times in total as that is the limiting division of the gardens and flowers. It's clear that for all layers, it must be $2R_{\lfloor n/2 \rfloor} + Cn \log n \leq 2 \left[R_{\lfloor n/4 \rfloor} + Cn\log n\right]$, so repeating this $\log_2 n$ times yields: $$R_n \leq 2^{\log_2n}R_1 + Cn\log^2 n = O(n \log^2 n).$$
\end{enumerate}
\end{solution}
\end{document}