\documentclass{article}

% Remember to also load the algostyle.sty file into your project.
\usepackage{algostyle}

% Insert new packages here.

\begin{document}
\begin{solution}
We begin by initialising a set of events based on the given intervals and sorting them. For some interval $I_i \in \mathcal{I}$, where $i = 1, \dots, n$ with $I_i = [a_i, b_i]$ where $a_i < b_i$, we set $a_i$ as a start event and $b_i$ as an end event. Repeat this for all intervals, making sure that each start and end event are corresponds to their original interval. This should be done in $O(n)$ time. If several intervals have end events at the exact same position as start events for other intervals, we need to append the start events first. This will take $O(n \log n)$.\\

Now, initialise an empty priority queue of available colours and a tracker for the maximum colours in the list, set to $0$. For each event $E$ in the sorted list, we use the following procedure.

\begin{itemize}
	\item If $E$ is a start event, check if there are any available colours in the list. If there are, assign it to the interval and remove it from the queue. This operation will be $O(\log n)$. Otherwise, increment the maximum tracker by one and add a new, distinct colour by some choice.
	\item If $E$ is an end event, add the corresponding colour back to the queue of available colours.
\end{itemize}

Since there are $2n$ events, and each check is $O(\log n)$, our process will be $O(n \log n)$. This means our algorithm will be $O(n \log n)$.
\end{solution}
\end{document}