\documentclass [12pt]{article}
\usepackage {amsmath}
\title{Typesetting Mathematics}
\author{Stephen Lerantges\\\textit{z5319858}}

\begin{document}
\maketitle
The area of a circle is given by $ A = \pi r ^2$.
The quadratic formula states that the solutions to \( ax ^2 + bx + c = 0\) are \[ x = \frac{-b \pm \sqrt{ b ^2 -4 ac }}{2 a }. \]
For example, if $a = 2$ , $ b = -5$ and $c = 3$ , then we have
\begin{align*}
x &= \frac{ -( -5) \pm \sqrt{( -5) ^2 - 4(2) (3) }}{2(2) }\\
&= \frac{5 \pm \sqrt{25 -24}}{4}\\
&= \frac{5 \pm 1}{4}\\
&= \frac{3}{2} \text { or } 1.
\end{align*}

Here's a fun equation:
\[\oint_C \vec{F} \cdot d\vec{r} = \iint_S (\nabla \times \vec{F}) \cdot d\vec{S}.\]
\end{document}