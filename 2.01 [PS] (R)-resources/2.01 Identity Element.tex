\documentclass{article}

% Remember to also load the algostyle.sty file into your project.
\usepackage{algostyle}

% Insert new packages here.

\begin{document}
\begin{question}
Let $A[1..n]$ be a sorted array of $n$ {\em distinct} integers. Some of these integers might be positive, negative, or zero.
\begin{enumerate}[label = (\alph*)]
    \item Describe an $O(\log n)$ algorithm to decide if there exist some index $i$ such that $A[i] = i$.

    {\bfseries Hint.} {\em Consider the array $B[i] = A[i] - i$.}

    \item Now, suppose we know that $A[1] > 0$. Describe an $O(1)$ algorithm to decide if there exist some index $i$ such that $A[i] = i$.

    {\bfseries Hint.} {\em Again, consider the array $B[i] = A[i] - i$.}
\end{enumerate}
\end{question}

\begin{rubric}
\begin{itemize}
    \item You should justify why your algorithm is correct and why they run in the allocated time complexities (or faster!).

    \item This task will form part of the portfolio.
    \item Ensure that your argument is clear and keep reworking your solutions until your lab demonstrator is happy with your work.
\end{itemize}
\end{rubric}

\begin{solution}
\begin{enumerate}[label = (\alph*)]
    \item Define a new array $B$, where $B[i] = A[i] - i$ for all $i = 1, \dots, n$. From here, we can simply perform a binary search starting from $\displaystyle i = \left \lfloor \dfrac{n}{2}  \right \rfloor$, looking for any element that equals zero. In the original array, if $A[i] = i$, then $B[i] = 0$. Since $A$ is sorted and made with distinct integers (always increasing), it means that $B$ is non-decreasing and hence can be searched in $O(\log n)$ time.

    \item Consider the array $B$ from the previous part. If $B[1] = 0$, then we must have $A[1] = 1$ and hence $i = 1$ is a valid index that meets the criteria. Now, $A$ must be increasing, meaning that $B$ is non-decreasing. If $B[1] \neq 0$, then $B[1] > 0$ since $A[1] > 0$, and so no preceeding term can be $0$. This means that, unless $B[1] = 0$, there cannot exist such an index.
\end{enumerate}
\end{solution}
\end{document}