\documentclass{article}

% Remember to also load the algostyle.sty file into your project.
\usepackage{algostyle}

% Insert new packages here.

\begin{document}
\begin{question}
Let $A[1..n]$ be an array of $n$ distinct and positive integers. Given an integer $x$, describe an $O(n \log n)$ algorithm to return the number of distinct pairs $(i, j)$ of indices where $1 \leq i < j \leq n$, satisfying:
\begin{itemize}
    \item $A[i] > A[j]$, and
    \item $A[i] + A[j] = x$.
\end{itemize}

{\bfseries Note.} {\em This was an assignment problem from COMP3121/9101, 23T2.}
\end{question}

\begin{rubric}
\begin{itemize}
    \item This task will form part of the portfolio.
    \item Ensure that your argument is clear and keep reworking your solutions until your lab demonstrator is happy with your work.
\end{itemize}
\end{rubric}

\begin{solution}
First, sort the array $A$ along with their original indices. We will call the sorted array $B$. This can be done in $O(n \log n)$ time using merge sort. For each element $B[i]$ in $B$, perform a binary search to find the index $j$ such that $B[j] = x - B[i]$ and $B[j]$'s original index is less than $B[i]$'s original index. Note that the elements in $A$ are distinct so there will not be any double-ups. This ensures that we only count pairs that satisfy both conditions: $A[i] + A[j] = x$ and $A[i] > A[j]$. During the binary search, keep a count of the number of valid pairs found. This count will be the final answer. The binary search for each element takes $O(\log n)$ time, so this step takes a total of $O(n \log n)$ time.
\end{solution}
\end{document}