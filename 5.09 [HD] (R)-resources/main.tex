\documentclass{article}

% Remember to also load the algostyle.sty file into your project.
\usepackage{algostyle}

% Insert new packages here.

\begin{document}
\begin{question}
Let $G = (V, E)$ be an undirected and unweighted graph. A {\em colouring} of $G$ is an assignment of vertices to colours such that no pair of adjacent vertices share the same colour. This problem is in $\mathsf{NP\text{-}C}$; however, we will come up with an exponential-time dynamic programming. We consider the $k\text{-}\compproblem{Colouring}$ problem, described below.

\begin{itemize}
    \item[] {\bfseries Instance.} An undirected and unweighted graph $G = (V, E)$.
    \item[] {\bfseries Task.} {\em Is there a colouring of $G$ using at most $k$ colours?}
\end{itemize}

\begin{enumerate}[label = (\alph*)]
    \item Describe a brute-force algorithm for $k\text{-}\compproblem{Colouring}$. What is the running time of such an algorithm?

    \item An {\em independent set} of $G$ is a subset of vertices $S \subseteq V$ such that no pair of vertices in $S$ are adjacent. How do independent sets of $G$ relate to colour classes of $G$?

    \item Prove that there exists an optimal $k$-colouring such that a colour class is a maximal independent set.

    We are now ready to describe an ``efficient'' algorithm.

    \item We enumerate over all subsets of $V$. For a subset $S \subseteq V$ of vertices, we define $G[S]$ to be the graph of $G$ {\em induced} by the vertex set $S$.
    
    \begin{enumerate}[label = (\roman*)]
        \item Let $\compproblem{OptColour}(S)$ denote the minimum $k$ such that $G[S]$ is $k$-colourable. Explain why
        \begin{align*}
            \compproblem{OptColour}(S) = 1 + \min\{\compproblem{OptColour}(S \setminus I) : I \text{ is a maximal indep. set in } G[S]\}.
        \end{align*}

        \item What is a suitable base case for this problem?

        \item What is the final solution, and what is the order of computation?

        \item On a graph with $n$ vertices, assume that all maximal independent sets can be generated in time $O^*(3^{n/3})$. Here, $O^*(\cdot)$ omits all polynomial factors; that is, $O^*(a^n) = O(p(n) \cdot a^n)$ where $p(n)$ is a polynomial in $n$.

        Show that $\compproblem{OptColour}(S)$ can be solved in time $O^*\left(3^{\lvert S \rvert /3}\right)$.

        \item Show that the running time is given by $O^*\left((1 + 3^{1/3})^n\right)$. This gives an approximately $O^*(2.4423^n)$ algorithm.

        {\bfseries Hint.} {\em The binomial theorem might come in handy.}
    \end{enumerate}
\end{enumerate}

{\bfseries Note.} {\em This is the algorithm described in \href{https://www.semanticscholar.org/paper/A-Note-on-the-Complexity-of-the-Chromatic-Number-Lawler/0742e3eac4efae7db8c0ac816223e2e4c51a93f6}{[Lawler, '76]}. This was the best known algorithm until a new inclusion-exclusion algorithm was introduced in 2006 by Bj\"orklund and Husfeldt that runs in $O^*(2^n)$; this is the currently best-known algorithm.}
\end{question}

\begin{solution}
\begin{enumerate}[label = (\alph*)]
    \item There are $k^{|V|}$ possible ways to assign $k$ colours to $|V|$ vertices, so we can generate each possible assignment and verify if it is valid by checking that each edge does not share a similarly colour vertex on each side. This will take $O(k^{|V|} \cdot |E|)$.

    \item Independent sets of $G$ represent potential colour allocations. Since no two vertices are adjacent, they can each be safely coloured the same colour. 

    \item Start by selecting a maximal independent set $M$ in $G$ and colour each of them the same. We can then remove each vertex in $M$ from $G$ and recursively find the next maximal independent set. Since each vertex is set a maximal and independent at the time of colouring, no further vertices can be added. Each colour class is a maximal and independent set within the context of the vertices remaining at each step. Therefore, at least one (and possible all) of the colour classes in an optimal $k$-colouring is a maximal independent set.

    \item
    \begin{enumerate}[label = (\roman*)]
        \item This algorithm describes the process from part (c), where we find the maximal independent subset, remove the coloured vertices and repeat the proecss, adding $1$ at each layer. This is optimal since the colourings chosen are based on the maximal possible selections of vertices and hence cannot be improved.

        \item We can define a simple and obvious base case of $\compproblem{OptColour}(\varnothing) = 0$.

        \item Our final solution will simply be $\compproblem{OptColour}(V)$.

        \item In the worst case, we will have a colouring of $k = n$ separate colours. We can calculate the time complexity of this as our worst case, being $\displaystyle \sum_{i=1}^{n} O^*(3^{i/3})$. Before we handle the sum, we must prove that  $O^*(n)  + O^*(m) = O^*(n + m)$. We'll let $f(n) = c_1 \cdot n^{k_1} \cdot g(n)$ and $h(m) = c_2 \cdot m^{k_2} \cdot p(m)$, where $g(n)$ and $p(m)$ dominate the growth.
\begin{itemize}
	\item If $m \approx n$, then we will take: $$f(n) + h(m) \approx (c_1 n^{k_1} + c_2 n^{k_2})\cdot 2^n,$$ and so in this case, $O^*(n) + O^*(m) = O^*(n+m)$.
	\item If $n$ and $m$ are not necessarily equal, then, without loss of generality, take $n$ to be the larger one. $2^m$ is then bounded by $2^n$ and thus the dominant term still grows faster than some exponential function in $n+m$, so $f(n) + f(m)$ can be bounded by a function of the form $c(n+m)^k \cdot 2^{n+m}$.
\end{itemize}

Since we know $O^*(n)  + O^*(m) = O^*(n + m)$, we can finish the proof by simplifying it as a geometric series:
\begin{align*}
	\sum_{i=1}^{n} O^*(3^{i/3}) &= O^* \left( \sum_{i=1}^{|S|} 3^{i/3}\right)\\
&= O^* \left(\dfrac{3^{1/3}}{3^{1/3}-1} (3^{|S|/3} - 1)\right)\\
&= O^* \left(3^{|S|/3}\right).
\end{align*}
        \item Recall that the complexity for each subset $S$ of vertices in terms of generating all maximal independent sets is $O^*(3^{|S|/3})$. The total complexity of solving the problem over all subsets $S$ can be expressed by considering all sizes $s$ of the subsets and applying the complexity for each size. That is,

$$\sum_{s=0}^n \binom{n}{k}O^*(3^{k/3}) = O^* \left( \sum_{s=0}^n \binom{n}{k} 3^{k/3} \right).$$

By the binomial theorem, where $\displaystyle (x+y)^n = \sum_{s=1}^n \binom{n}{s}x^{n-s}y^{s}$, we see that this expression simplifies to $O^*\left( (1 + 3^{1/3})^n\right)$.


    \end{enumerate}
\end{enumerate}
\end{solution}
\end{document}