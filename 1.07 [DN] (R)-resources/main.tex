\documentclass{article}

% Remember to also load the algostyle.sty file into your project.
\usepackage{algostyle}

% Insert new packages here.

\begin{document}
\begin{question}
Let $G = (V, E)$ be a directed and acyclic graph. We will refer to these as DAGs, and they will become important structures later on. In this problem, we will prove some basic results about DAGs.

\begin{enumerate}[label = (\alph*)]
    \item Show that there exist a vertex in $G$ that has no incoming edges. We refer to these as {\em source} vertices.

    \item Show that there exist a vertex in $G$ that has no outgoing edges. We refer to these as {\em sink} vertices.

    \item \label{topsort} Show that there exist a mapping $f : V \to \{1, \dots, n\}$ such that $(u, v) \in E$ implies that $f(u) \leq f(v)$.

    Another way to interpret this is that this mapping {\em respects} the direction of the edges. We refer to this ordering as a {\em topological order} of $G$. You will later see that we can {\em sort} these vertices by its topological order; this is called a {\em topological sort}.
\end{enumerate}
\end{question}

\begin{solution}
\begin{enumerate}[label = (\alph*)]
    \item Suppose that every edge in our graph does have at least an incoming edge. If we pick a random starting vertex, we will be able to find a predecessor and thus can move to that vertex and continue the process. Since our graph is finite, we must at some point repeat a vertex since we will always be able to find a predecessor. However, this would imply the existence of a cycle. Therefore, by contradiction, our initial assumption is false, and so there must exist a vertex in $G$ with no incoming edges.

    \item Similar to part (a), we will assume that there is indeed at least one outgoing edge for each vertex. Starting from some vertex, we can always find a new vertex to travel to along the directed edge. Since the number of vertices in the graph is finite, it implies that at some point we must repeat a vertex at some point, proving the existence of at least one cycle. Therefore, by contradiction, our initial assumption is false, and so there must exist a vertex in $G$ with no outgoing edges.

    \item We can prove the existence of a mapping $f : V \to {1, \dots, n}$ for a DAG with $n$ vertices using mathematical induction.\\
    
    \textbf{Base Case:} When $n = 1$, there is only one vertex in the graph. We can define the mapping $f : V \to {1}$ such that $f(v) = 1$ for the single vertex $v$. This trivially satisfies the condition that $(u, v) \in E$ implies $f(u) \leq f(v)$, since there are no edges in the graph.\\
    
    \textbf{Inductive Step:} Assume the statement holds for a DAG with $n$ vertices. For a DAG $G'$ with $n+1$ vertices, remove a source vertex $v$ to obtain a smaller DAG $G''$ with $n$ vertices. By the induction hypothesis, there exists a mapping $f'' : V(G'') \to \{1, \dots, n\}$ that satisfies the condition. Extend this mapping to $G'$ by defining $f' : V(G') \to \{1, \dots, n+1\}$ as $f'(w) = f''(w)$ for $w \in V(G'')$ and $f'(v) = n+1$. This mapping respects the direction of the edges in $G'$.\\
    
    By the principle of mathematical induction, the statement holds for any DAG with $n$ vertices.
\end{enumerate}
\end{solution}
\end{document}