\documentclass{article}

% Remember to also load the algostyle.sty file into your project.
\usepackage{algostyle}

% Insert new packages here.

\begin{document}
\begin{question}
Let $G = (V, E, w)$ be a directed and weighted graph with positive edge weights $w(e) > 0$ for each edge $e \in E$. For a pair of vertices $u, v \in V$, there may be {\em multiple} shortest paths from $u$ to $v$. Let $\Pi_{u, v}$ denote all such paths. In other words, for a pair of vertices $u$ and $v$, $\Pi_{u, v}$ is the set of all shortest paths from $u$ to $v$.

A vertex $x$ is called {\em useful} if $x$ lies on {\em any} path in $\Pi_{u, v}$. Given the graph $G$ and a pair of vertices $u, v \in V$, describe an $O(m \log n)$ algorithm to return all useful vertices.
\end{question}

\begin{solution}
To find all useful vertices in $G$ defined on path $\Pi_{u,v}$, we first run Dijkstra's algorithm from $u$ to get shortest path distances. We must now repeat this for vertex $v$, however we must create a new graph with all edges reversed in order to be able to run Dijkstra's algorithm from this vertex. Then, for each vertex $x \in V$, if the distance to $x$ from $u$ plus the distance distance from $v$ to $x$ (in the reversed graph) equals the shortest distance between $u$ and $v$, then the point is useful. Append each of these vertices to a list and return it. This approach has a time complexity of $O(m \log n + n)$ as Dijkstra's algorithm is in $O(m \log n)$ time, and we may visit all vertices $n$.
\end{solution}
\end{document}