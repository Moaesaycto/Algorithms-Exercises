\documentclass{article}

% Remember to also load the algostyle.sty file into your project.
\usepackage{algostyle}

% Insert new packages here.

\begin{document}
\begin{question}
What is your view of the $\mathsf P$ vs $\mathsf{NP}$ problem?
\end{question}

\begin{solution}

While many computer scientists lean towards believing that  $P\neq NP$, I find it compelling to consider both possibilities. One of the most significant insights from my studies at UNSW is the importance of not viewing computer science and mathematics merely as collections of numbers and equations but as fields rich with philosophical depth that can redefine our understanding of human decision-making based on available information. If it turns out  $P=NP$, it could instill hope that the problems we face are solvable, providing reassurance that our efforts in optimising key algorithms could indeed bear fruit. I personally believe that even the mere possibility of a problem being solvable could spark a massive and rapid wave of innovation in technology, medicine, and science, even if the algorithms themselves remain somewhat elusive.

In terms of the nature of our universe, $P=NP$ might suggest a disheartening simplicity in how our brains function. In the realm of decision-making, this revelation could challenge our notions of creativity, as many forms of abstract expression might be reduced to basic processes. The implications of $P=NP$ might also suggest a deterministic universe, where solutions to complex problems like the Navier–Stokes equations, or the fundamental behaviors of particles, could be simplified to mere polynomial solutions. While the potential for extreme data breaches and hacking is alarming, I find the prospect of questioning the intricacies of human thought even more daunting.

However, this is precisely why I lean towards believing $P \neq NP$. As an undergraduate encountering these concepts for the first time, my experience might not yet support a mathematical argument, but I believe that the groundbreaking abstract ideas that have propelled human technology—such as Newton's invention of calculus, Alan Turing's work, and Einstein's theory of special relativity—could not have emerged from simple, efficient algorithms. True art emanates from a place that no polynomial time solution can predict, and I am inclined to maintain this stance unless proven otherwise.
\end{solution}
\end{document}