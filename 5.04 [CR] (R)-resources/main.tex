\documentclass{article}

% Remember to also load the algostyle.sty file into your project.
\usepackage{algostyle}

% Insert new packages here.

\begin{document}
\begin{question}
You are given two lists of $n$ points, one list $P = [p_1, \dots, p_n]$ lies on the line $y = 0$ and the other list $Q = [q_1, \dots, q_n]$ lies on the line $y = 1$. We construct $n$ line segments by connecting $p_i$ to $q_i$ for each $i = 1, \dots, n$. You may assume that the numbers in $P$ are distinct and the numbers in $Q$ are also distinct. Describe an $O(n^2)$ algorithm to return the size of the largest subset $L$ of line segments such that no pair of line segments in $L$ intersect.

For example, the following instance
\begin{center}
    \begin{tikzpicture}
        % Line segments.
        \draw[black, thick, latex-latex] (0, -0.5) -- (7, -0.5);
        \draw[black, thick, latex-latex] (0, 1) -- (7, 1);
    \definecolor{yellowgreen}{rgb} {0.76,0.83,0}

        \node[circle, color=red!60!black, fill, inner sep = 2pt, outer sep = 0pt, label = below:{$p_1$}] at (1, -0.5) (p1) {};
        \node[circle, color=babyblueeyes, fill, inner sep = 2pt, outer sep = 0pt, label = below:{$p_2$}] at (2.25, -0.5) (p2) {};
        \node[circle, color=black , fill, inner sep = 2pt, outer sep = 0pt, label = below:{$p_3$}] at (5, -0.5) (p3) {};
        \node[circle, color=yellow!60!black, fill, inner sep = 2pt, outer sep = 0pt, label = below:{$p_4$}] at (4, -0.5) (p4) {};

        \node[circle, color=red!60!black, fill, inner sep = 2pt, outer sep = 0pt, label = above:{$q_1$}] at (6, 1) (q1) {};
        \node[circle, color=babyblueeyes, fill, inner sep = 2pt, outer sep = 0pt, label = above:{$q_2$}] at (3, 1) (q2) {};
        \node[circle, color=black , fill, inner sep = 2pt, outer sep = 0pt, label = above:{$q_3$}] at (0.5, 1) (q3) {};
        \node[circle, color=yellow!60!black , fill, inner sep = 2pt, outer sep = 0pt, label = above:{$q_4$}] at (4, 1) (q4) {};

        % Draw line segments.
        \foreach \x in {1, ..., 4}
            \draw[thick] (p\x) -- (q\x);
    \end{tikzpicture}
\end{center}
should return 2. The largest subset of pairwise-nonintersecting line segments is $\{(p_2, q_2), (p_4, q_4)\}$.

{\bfseries Note.} {\em This is slightly different to the divide-and-conquer problem, which asked for the number of intersections of line segments. This problem asks for the size of the largest subset of {\bfseries non-intersecting} line segments.}

There are at least two approaches.

{\bfseries Approach 1.} {\em You can directly use dynamic programming...}

{\bfseries Approach 2.} {\em Find a way to reduce this to the longest increasing subsequence problem.}
\end{question}

\begin{solution}
We first begin by sorting the points according to their starting points $P$, keeping track of their positions in $Q$ during the sort. This sorting will take $O(n \log n)$. From here, we simply need to find the longest increasing subsequence in $Q$. A non-intersecting subset of segments corresponds to an increasing sequence on end points in $Q$ as otherwise, there would exist an intersection point (where $p_i < p_j$ but $q_i > q_j$, there must be an intersection).\\

From here, the application is a standard usage of Dynamic Programming, where we make a list of $n$ numbers $DP$, where $DP[i]$ represents the length of the LIS ending at $q_i$. For each $i$ from $1$ to $n$, we run for each $j$ from $0$ to $i-1$, if $q_j < q_i$, we update $DP[i] = \max \{DP[i], DP[j] + 1\}$. The maximum value in $DP$ will correspond to our desired answer. This is clearly done in $O(n^2)$ time.
\end{solution}
\end{document}