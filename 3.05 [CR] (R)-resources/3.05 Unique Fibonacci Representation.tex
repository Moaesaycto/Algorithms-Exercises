\documentclass{article}

% Remember to also load the algostyle.sty file into your project.
\usepackage{algostyle}

% Insert new packages here.

\begin{document}
\begin{solution}
\begin{enumerate}[label=(\alph*)]
\item Let $i $ be the current value of our algorithm at some recursive step. In the case where the algorithm does not stop, we set $i \to i - F_k$, where $F_k$ is the maximum Fibonacci term such that $F_k \leq i$. Consider now if $F_{k-1}$ was the largest Fibonacci number that fits in $n - F_k$. This would imply
\begin{align*}
	n - F_k - F_{k-1} &\geq 0\\
	n - (F_k + F_{k-1}) & \geq 0\\
	n - F_{k+1} &\geq 0\\
	n &\geq F_{k+1}.
\end{align*}

This, however, is a contradiction as it would imply that $F_{k+1}$ fits into $n$, and so $F_k$ is not the largest. Thus, the next highest must be at most $F_{k-2}$. Thus, our sequence would be a sum of non-consecutive Fibonacci numbers. Each iteration will never reach a negative point, but it will always be decreasing. Once we reach $i = 0$, $1$ or $2$, the algorithm stops. Thus, this algorithm works.

\item Any sum of distinct, non-consecutive Fibonacci numbers that are at most $F_i$ is less than $F_{i+1}$. This can be proven by induction as follows.

\begin{itemize}
	\item \emph{Base case}: For $i = 2$, it's clear that the only set to consider would be $\{F_2\} = \{F_1\}$, who's sum is $1 < 2 = F_3$.
	\item \emph{Inductive step}: Assume our statement holds for all sets of distinct, non-consecutive Fibonacci numbers whose largest member is $F_i$. That is, the sum of all members in this set is less than $F_{i+1}$. Consider now any non-empty set $S$ of distinct, non-consecutive  Fibonacci numbers whose largest member is $F_{i+1}$. Let $T$ be a set of distinct, non-consecutive Fibonacci numbers with a largest member $F_{i-1}$ (since the members are non-consecutive) such that $S = \{F_{i+1}\} \cup T$. Therefore, by our inductive hypothesis, $$\sum_{s \in S} s = F_{i+1} + \sum_{t \in T} t < F_{i+1} + F_i = F_{i+2}.$$
\end{itemize}
By the principle of mathematical induction, our statement has been proven true. Now, consider two non-empty sets $S_1$ and $S_2$ of distinct, non-consecutive Fibonacci numbers which have the same total sum. If we remove any of the elements in common from these sets, the resulting sets must be equal in terms of their sums as well - because we are removing equal values from both sets. Assume that these new sets, say $S_1'$ and $S_2'$, are non-empty. The largest terms in each of them must then not be equal since $S_1' \cap S_2' = \varnothing$. Without loss of generality, assume that $F_{s_1} < F_{s_2}$, where $F_{s_1}$ and $F_{s_2}$ are the largest terms in $S_1'$ and $S_2'$ respectively. It must be the case that, by our induction proof, \[\sum_{x\in S_1'} x < F_{s_1+1} \leq F_{s_2}.\] But we know that $\displaystyle \sum_{x \in S_2'} x \geq F_{s_2}$, thus we have a contradiction and so $S_1' = S_2' = \varnothing $.\\

Ultimately, we have proven that $S_1$ and $S_2$ cannot have any differences, so the algorithm from part (a) must produce a unique output.
\end{enumerate}
\end{solution}
\end{document}