\documentclass{article}

% Remember to also load the algostyle.sty file into your project.
\usepackage{algostyle}

% Insert new packages here.

\begin{document}
\begin{question}
You are given a long piece of stick of length $L$. You want to cut the stick into {\em exactly} $n$ places along its length, where the $i$th place to cut occurs at position $A[i]$. Since sticks of larger size require more power, cutting a stick of length $X$ requires $X$ units of work.

\begin{enumerate}[label = (\alph*)]
    \item Given the stick of length $L$ and the positions of the $n$ places to cut $A[1..n]$, describe an $O(n^3)$ algorithm to compute the minimum number of units of work to cut the stick into $n$ pieces.

    \item Describe an $O(n^2)$ algorithm to compute the minimum number of units of work to cut the stick into $n$ pieces.

    {\bfseries Hint.} {\em What can we optimise here?}
\end{enumerate}
\end{question}

\begin{solution}
\begin{enumerate}[label = (\alph*)]
    \item We initialise a Dynamic Programming table $DP$, which will represent a $(n+2) \times (n+2)$ array. $DP[i][j]$ represents the minimum units of work required to cut the stick from position $i$ to position $j$, where we also consider $0$ and $L$ as potential cutting positions. Clearly, for all $i = 0, \dots, L$, we have $DP[i][i+1] = DP[i][i] = 0$. For each possible length, $l$, of the stick segment ($2$ to $n+1$), we run a second loop for each position $i$, we define $j = i + l - 1$, and then for each cut position $k$ between $i$ and $j$, we calculate the minimum work required, updating $DP[i][j]$ respectively. We are running three nested loops, so we have a time complexity of $O(n^3)$. Our final result will simply be $DP[0][n+1]$.

    \item To optimise our solution, we should consider the factthat the optimal way to cut a segment does not depend on the cuts made outside of that segment. 
\end{enumerate}
\end{solution}
\end{document}